\documentclass{muformallab}

\begin{document}

  \newcommand{\results}[6]{
    \begin{center}
      \begin{tabular}{c c}
        Operation & Time (ms) \\
        \hline
        Random Add & #1 \\
        Random Find & #2 \\
        Random Remove & #3 \\
        Sequential Add & #4 \\
        Sequential Find & #5 \\
        Sequential Remove & #6 \\
      \end{tabular}
    \end{center}
  }

  \begin{titlepage}
    \vspace*{\stretch{1.0}}
    \begin{center}
      \Large\textbf{Data Structures}\\
      \large\textit{Some are faster than others}\\
      \large\text{Nathan J. Mara}\\
      \large\text{2014-11-19}\\
      \large\text{CSE 274}
    \end{center}
    \vspace*{\stretch{2.0}}
  \end{titlepage}

  \section{Binary Search Trees}

  A Binary Search Tree is a Binary Tree in which each node is larger than
  the nodes in its left subtree and smaller than the nodes in its right
  subtree. It is also sometimes called an Ordered Binary Tree.

  \subsection{The Good}

  In theory, non-balancing Binary Search Trees should perform
  find/add/remove operations in $\Theta \left( log \left( n \right)
  \right)$ time when balanced. This is due to the fact that to perform any
  of these functions, you must visit at most $tree.height$ items, and
  a balanced Binary Search Tree has height $log \left( n \right)$. When
  traversing a Binary Search Tree, you can use the ``small on left, large
  on right'' property of Binary Search trees to compare the expected
  element to the current one, and it will only have to visit $\Theta
  \left( log \left( n \right) \right)$ items.

  \subsection{The Bad}

  In practice, it is very easy to create a Binary Search Tree that is
  unbalanced, and as a result has a complexity of $\Theta \left(
  n \right)$. In fact, of all of the data structures tested, the
  non-balancing Binary Search Tree had the worst performance scores.

  \subsection{The Numbers}

  \results{4164.25}{0.25}{0.4}{6357.2}{5703.1}{2.05}

  \section{Red-Black Trees}

  A Red-Black Tree is a kind of Binary Search Tree that balances itself as
  elements are added to it.

  \subsection{The Good}

  The Red-Black Tree is significantly faster than the Binary Search Tree
  in testing because it is self balancing. This means that as elements are
  added to the Red-Black Tree, it corrects for the order of the elements
  automatically in an attempt to keep the complexity as low as possible.
  The Red Black Tree has a maximum complexity of $\Theta \left( 2 log
  \left( n \right) \right)$.

  \subsection{The Bad}

  Red-Black Trees under-perform both types of Hash Tables tested here in
  adding $50,000$ random elements. This is probably due to the reshaping
  of the tree that the self-balancing algorithm does. Red-Black trees take
  up more space than non-balancing Binary Search Trees because each node
  also has a \textit{color} property associated with it.

  \subsection{The Numbers}

  \results {13.05} {0.15} {0.25} {11.1} {4.75} {8}

  \section{Chained Hash Tables}

  A Chained Hash Table is a hash table where every element is stored on
  a secondary ``chain'' rather than the primary backing array.

  \subsection{The Good}

  In testing, the Chained Hash Table tested far better than the Binary
  Search Tree in random add, sequential add and sequential find. It was
  slower than or equal to the Linear Hash Table in all tests, and
  comparable to Red-Black Tree in all tests with the exception of being
  slower in sequential add and sequential find.

  \subsection{The Bad}

  The Chained Hash Table is very slow on the sequential add test. It takes
  over $42$ ms compared to $~3$ ms for Linear Hash Table. When new
  elements are added to the Chained Hash Table, it has to not only resize
  its backing array, but create more ``chain'' objects and fill those.
  Adding objects to a Chained Hash Table has best case complexity $\Theta
  \left( hash \left( n \right) \right)$ and worst case complexity $\Theta
  \left( n \right)$.

  \subsection{The Numbers}

  \results {9.95} {0.6} {0.4} {42.4} {9.2} {7.45}

  \section{Linear Hash Tables}

  \subsection{The Good}

  \subsection{The Bad}

  \subsection{The Numbers}

  \results {1.6} {0.55} {0.45} {3.6} {0.95} {2.85}

\end{document}
